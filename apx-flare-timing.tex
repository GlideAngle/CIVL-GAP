% !TEX TS-program = xelatex
% !TEX encoding = UTF-8 Unicode
\documentclass[gap.tex]{subfiles}
\begin{document}
The \href{https://github.com/BlockScope/flare-timing}{flare-timing}
implementation of GAP is a collection of apps, one for each step of the scoring
process, see Fig~\ref{fig:flare-timing}. Not being a single app that can store
intermediate results in memory, its workings and final results are written to
file with enough detail and evidence so that everything can be checked by hand.

\subsection{Scoring Sequence of Steps}

\begin{figure}[!ht]
    \centering
    \begin{tikzcd}
    \textit{comp}
    \texttt{.fsdb}
    \arrow[d, "\text{extract input}"]
    &
    \\
    \textit{comp}
    \texttt{.comp-input.yaml}
    \arrow[d, "\text{cross zone}"]
    &
    \textit{task}
    \texttt{/}
    \textit{pilot}
    \texttt{.kml}
    \arrow[dl, "\text{cross zone}"]
    \\
    \textit{comp}
    \texttt{.cross-zone.yaml}
    \arrow[d, "\text{tag zone}"] & \\
    \textit{comp}
    \texttt{.tag-zone.yaml}
    \arrow[d, "\text{align time}"]  & \\
    \texttt{.flare-timing/align-time/task-n/}
    \textit{pilot}
    \texttt{.csv}
    \arrow[d, "\text{discard further}"]  & \\
    \texttt{.flare-timing/discard-further/task-n/}
    \textit{pilot}
    \texttt{.csv}
    \arrow[d, "\text{mask track}"]  & \\
    \textit{comp}
    \texttt{.mask-track.yaml}
    \arrow[d, "\text{land out}"]  & \\
    \textit{comp}
    \texttt{.land-out.yaml}
    \arrow[d, "\text{gap point}"]  & \\
    \textit{comp}
    \texttt{.gap-point.yaml} &
\end{tikzcd}

    \caption{Scoring as numbered steps from {\color{blue}original inputs} to
    {\color{csv}\texttt{*.csv}} or \texttt{*.yaml} outputs.}
    \label{fig:flare-timing}
\end{figure}

Starting with an \texttt{*.fsdb} comp and related \texttt{*.igc} or
\texttt{*.kml} track logs, the scoring process proceeds along this 
sequence of steps where outputs from earlier steps may be referred to later on;
\begin{enumerate}
    \item
        Extract the inputs with
        \href{https://github.com/BlockScope/flare-timing/tree/master/flare-timing/prod-apps/extract-input}{extract-input}.
    \item
        Trace the route of the shortest path to fly a task with
        \href{https://github.com/BlockScope/flare-timing/tree/master/flare-timing/prod-apps/task-length}{task-length}.
    \item
        Find pairs of fixes crossing over zones with
        \href{https://github.com/BlockScope/flare-timing/tree/master/flare-timing/prod-apps/cross-zone}{cross-zone}.
    \item
        Interpolate between crossing fixes for the time and place where a track
        tags a zone with
        \href{https://github.com/BlockScope/flare-timing/tree/master/flare-timing/prod-apps/tag-zone}{tag-zone}.
    \item
        Index fixes from the time of first crossing with
        \href{https://github.com/BlockScope/flare-timing/tree/master/flare-timing/prod-apps/align-time}{align-time}.
    \item
        Discard fixes that get further from goal and note leading area with
        \href{https://github.com/BlockScope/flare-timing/tree/master/flare-timing/prod-apps/discard-further}{discard-further}.
    \item
        Mask a task over its tracklogs with
        \href{https://github.com/BlockScope/flare-timing/tree/master/flare-timing/prod-apps/mask-track}{mask-track}.
    \item
        Group and count land outs with
        \href{https://github.com/BlockScope/flare-timing/tree/master/flare-timing/prod-apps/land-out}{land-out}.
    \item
        Score the competition with
        \href{https://github.com/BlockScope/flare-timing/tree/master/flare-timing/prod-apps/gap-point}{gap-point}.  
\end{enumerate}

\newpage
\subsection{Extracting Inputs}
In the \texttt{*.fsdb} FS keeps both inputs and outputs. We're only interested
in a subset of the input data, just enough to do the scoring;

\begin{description}
    \item[Competition] id, name, location, dates and UTC offset.
    \item[Nominal] launch, goal, time, distance and minimal distance.
    \item[Task] name and type of task, zones, speed section, start gates and pilots.
    \item[Zone] name, latitude, longitude and altitude and radius if a cylinder
    \item[Pilot] name and either absentee status or track log file name.
\end{description}

Something to be aware of when parsing the \texttt{XML} of the \texttt{*.fsdb}
is that attributes may be missing and in that case we'll have to infer the
defaults used by FS. This is done by looking at the source code of FS as there
is no schema for the \texttt{XML} that could also be used to set default
values.

\begin{lstlisting}[language=XML, caption={Overall *.fsdb structure, filtered for input.}]
<Fs>
  <FsCompetition id="7592" name="2012 Hang Gliding Pre-World Forbes" location="Forbes, Australia"
      from="2012-01-05" to="2012-01-14" utc_offset="11">
    <!-- Nominals are set once for a competition but beware, they are repeated per task. -->
    <FsScoreFormula min_dist="5" nom_dist="80" nom_time="2" nom_goal="0.2" />
    <FsParticipants>
      <FsParticipant id="23" name="Gerolf Heinrichs" />
      <FsParticipant id="106" name="Adam Parer" />
    </FsParticipants>
      <!-- Flags on how to score are also set for the competition but pick them up from the task. -->
      <FsTask name="Day 8" tracklog_folder="Tracklogs\day 8">
        <FsScoreFormula use_distance_points="1" use_time_points="1" use_departure_points="0" use_leading_points="1" use_arrival_position_points="1" use_arrival_time_points="0" />
        <FsTaskDefinition ss="2" es="5" goal="LINE" groundstart="0">
          <!-- Not shown here but each FsTurnpoint has open and close attributes. -->
          <FsTurnpoint id="FORBES" lat="-33.36137" lon="147.93207" radius="100" />
          <FsTurnpoint id="FORBES" lat="-33.36137" lon="147.93207" radius="10000" />
          <FsTurnpoint id="MARSDE" lat="-33.75343" lon="147.52865" radius="5000" />
          <FsTurnpoint id="YARRAB" lat="-33.12908" lon="147.57323" radius="400" />
          <FsTurnpoint id="DAY8GO" lat="-33.361" lon="147.9315" radius="400" />
          <!-- This was an elapsed time task so no start gates. -->
        </FsTaskDefinition>
        <FsTaskState stop_time="2012-01-14T17:22:00+11:00" />
        <FsParticipants>
          <!-- An empty participant element is a pilot absent from the task. -->
          <FsParticipant id="106" />
          <FsParticipant id="23">
            <FsFlightData tracklog_filename="Gerolf_Heinrichs.20120114-100859.6405.23.kml" />
          </FsParticipant>
        </FsParticipants>
      </FsTask>
    </FsTasks>
  </FsCompetition>
</Fs>
\end{lstlisting}

\subsection{Finding Zone Crossings}

First off, before we can determine if any zones have been crossed we'll have to
decide how to tell which parts of a track log are flown and which are walked or
driven in the retrieve car, possibly even back to goal.\footnote{Some pilots'
track logs will have initial values way off from the location of the device.
I suspect that the GPS logger is remembering the position it had when last
turned off, most likely at the end of yesterday's flight, somewhere near where
the pilot landed that day. Until the GPS receiver gets a satellite fix and can
compute the current position the stale, last known, position gets logged. This
means that a pilot may turn on their instrument inside the start circle but
their tracklog will start outside of it.}

To work out when a pilot is flying, select the longest run of fixes that are
not the same allowing for some stickiness when the GPS loses signal. For
example we might consider within ± 1m altitude or within ± 1/10,000th of
a degree of latitude or longitude to be in the same location and not recorded
during flight.

\begin{lstlisting}[caption={Which fixes are considered flown, \texttt{flying} nodes of \texttt{*.cross-zone.yaml}.}]
flying:
  - - Gerolf Heinrichs
    - loggedFixes: 4786
      flyingFixes:
      - 293
      - 4775
      loggedSeconds: 19140
      flyingSeconds:
      - 1172
      - 19100
      loggedTimes:
      - 2012-01-14T02:00:05Z
      - 2012-01-14T07:19:05Z
      flyingTimes:
      - 2012-01-14T02:19:37Z
      - 2012-01-14T07:18:25Z
\end{lstlisting}

\subsection{Interpolating Zone Tagging}
\subsection{Time Aligning Flights}
\subsection{Incrementally Closer to Goal}
\subsection{Masking Task over Track}
\subsection{Assessing Difficulty}
\subsection{Collating Scores}

\end{document}
