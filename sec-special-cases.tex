% !TEX TS-program = xelatex
% !TEX encoding = UTF-8 Unicode
\documentclass[gap.tex]{subfiles}
\begin{document}
\label{sec:special-cases}
\subsection{ESS but not goal}
In a task where ESS and goal are not identical, a pilot may reach ESS, but not
goal.  Reaching goal is seen as “validating” one’s speed section performance.
A pilot who does not reach goal after reaching ESS will lose his time points.
He will only score distance points for the distance actually covered, and
leading points. This is seen as a safety measure, since it encourages pilots to
plan their final glide to ESS with enough altitude to safely reach goal. This
discourages high-speed final glides low to the ground.
\begin{align*}
    \forall p : p \in PilotsLandedBetweenESSandGoal : TotalScore_p &= \\
    DistancePoints_p + LeadingPoints_p + 0 * (TimePoints_p + ArrivalPoints_p)
\end{align*}

\subsection{Early start}
\label{sec:early-start}
An early start occurs if a pilot’s last SSS cylinder boundary crossing in start
direction (enter or exit) occurred before the first (or only) start gate time.

\begin{pg}
In paragliding, pilots who perform an early start are only scored for the
distance between the launch point and the SSS cylinder, as calculated when
determining the complete task distance (see ~\ref{sec:task-distance}).
\end{pg}

\begin{hg}
In hang-gliding, the so-called “Jump the Gun”-rule applies: If the early start
occurred within a time that is close to the first (or only) start gate time,
the pilot is scored for his complete flight, but a penalty is then applied to
his total score.

The penalty calculation is based on two values X and Y, which are set in S7A,
but can be changed at the task briefing (presumably by the meet director and/or
the task committee). For each X seconds a pilot starts early, he incurs
a 1 point penalty, up to a maximum of Y seconds. If a pilot starts more than
Y seconds early, he will only be scored for minimum distance.
\begin{align*}
    X_{default} &= 2 \\
    Y_{default} &= 300 \\
    timeDiff_p &= firstStartGateTime - lastStartTime_p
\end{align*}
\begin{align*}
    timeDiff_p \leq 0 : jumpTheGunPenalty_p &= 0 \\
    \\
    timeDiff_p > Y : jumpTheGunPenalty_p &= 0,\\
    totalScore_p = scoreForMinDistance \\
    \\
    0 < timeDiff_p \leq Y : jumpTheGunPenalty_p &= \frac{timeDiff_p}{X} \\
    totalScore_p = \max(totalScore_p - jumpTheGunPenalty_p, scoreForMinDistance)
\end{align*}
\end{hg}

\subsection{Stopped tasks}
\label{sec:stopped-tasks}
\subsubsection{Stop task time}
\label{sec:stop-task-time}
A task can be stopped at any time by the meet director. The time when a stop
was announced for the first time is the “task stop announcement time”. This
time must be recorded to score the task appropriately. For scoring purposes,
a “task stop” time is calculated. This is the time which determines whether
a task will be scored at all. Pilots’ flight will only be scored up to this
task stop time.

\begin{hg}
In hang-gliding, stopped tasks are “scored back” by a time that is determined
by the number of start gates and the start gate interval: The task stop time is
one start gate interval, or 15 minutes in case of a single start gate, before
the task stop announcement time.
\begin{align*}
    numberOfStartGates = 1 : taskStopTime &= taskStopAnnouncementTime - 15min \\
    numberOfStartGates > 1 : taskStopTime &= taskStopAnnouncementTime - startGateInterval \\
\end{align*}
\end{hg}

\begin{pg}
In paragliding the score-back time is set as part of the competition parameters
(see section ~\ref{sec:score-back-time}).
\begin{align*}
    taskStopTime &= taskStopAnnouncementTime - competitionScoreBackTime \\
\end{align*}
\end{pg}

\subsubsection{Requirements to score a stopped task}
For a stopped task to be scored, it must fulfil certain requirements, which
differ between the two disciplines:

\begin{hg}
In hang gliding, a stopped task can only be scored if either a pilot reached
goal or the race had been going on for a certain minimum time. The minimum time
depends on whether the competition is the Women’s World Championship or not.
The race start time is defined as the time when the first valid start was taken
by a competition pilot.
\begin{align*}
    typeOfCompetition = Women's : minimumTime &= 60min \\
    typeOfCompetition \neq Women's : minimumTime &= 90min \\
    a \land b : taskValidity &= 0 \\
    \text{where} \\
    a = taskStopTime - timeOfFirstStart < minimumTime \\
    b = numberOfPilotsInGoal(taskStopTime) = 0
\end{align*}
Note that this rule is currently not enforced by FS: The decision whether
a stopped task is cancelled or scored must be taken by the score keeper.
\end{hg}

\begin{pg}
In paragliding, a stopped task will be scored if the flying time was one hour
or more. For Race to Goal tasks, this means that the Task Stop Time must be one
hour or more after the race start time. For all other tasks, in order for them
to be scored, the task stop time must be one hour or more after the last pilot
started.
\begin{align*}
    minimumTime &= 60min \\
    \\
    typeOfTask = RaceToGoal \land numberOfStartGates = 1 : \\
    taskStopTime - startTime < minimumTime : \\
    taskValidity &= 0 \\
    \\
    typeOfTask \neq RaceToGoal \lor numberOfStartGates > 1 : \\
    taskStopTime - \max(\forall p : p \in StartedPilots . startTime_p) < minimumTime : \\
    taskValidity &= 0
\end{align*}
\end{pg}

\subsubsection{Stopped task validity}
For stopped tasks, an additional validity value, the Stopped Task Validity, is
calculated and applied to the Task Validity.
\[ DayQuality_{stopped} = LaunchValidity * DistanceValidity * TimeValidity * StoppedTaskValidity \]
Stopped Task Validity is calculated taking into account the task distance, the
flown distances of all pilots, the number of launched pilots and the number of
pilots still flying at the time when the task was stopped.
\begin{align*}
    NumberOfPilotsReachedESS > 0 : StoppedTaskValidity &= 1 \\
    NumberOfPilotsReachedESS = 0 : StoppedTaskValidity &= \min(1, a + b^3) \\
\end{align*}
\begin{align*}
    a &= \sqrt{\frac{BestDistFlown - avg(\forall i : DistFlown_i)}{DistLaunchToESS - BestDistFlown + 1} * \sqrt{\frac{stddev(\forall i : DistFlown)}{5}}} \\
    b &= \frac{NumPilotsLandedBeforeStopTime}{NumPilotsLaunched}
\end{align*}

\subsubsection{Scored time window}
For stopped Race to Goal tasks with a single start gate, scoring considers the
same time window for all pilots: The time between the race start and the task
stop time.
\begin{align*}
    typeOfTask = RaceToGoal \land numberOfStartGates = 1 : \forall p : p \in StartedPilots : \\
    scoreTimeWindow_p &= (startTime, taskStopTime)
\end{align*}
For stopped Race to Goal tasks with multiple start gates, as well as Elapsed
Time races, must be treated slightly differently: Only the time window
available to the last pilot started is considered for scoring. This time window
is defined as the amount of time t between the last pilot’s start and the task
stop time. For all pilots, only this time t after their respective start is
considered for scoring.
\begin{align*}
    typeOfTask \neq RaceToGoal \lor numberOfStartGates > 1 : \\
    scoreTime = taskStopTime - \max(\forall p : p \in StartedPilots : startTime_p) : \\
    \forall p : p \in StartedPilots : \\ 
    scoreTimeWindow_p &= (startTime_p, startTime_p + scoreTime)
\end{align*}
This means that if the last pilot started and then flew for, for example, 75
minutes until the task was stopped, all tracks are only scored for the first 75
minutes each pilot flew after taking their respective start.

\subsubsection{Time points for pilots at or after ESS}
Pilots who were at a position between ESS and goal at the task stop time will
be scored for their complete flight, including the portion flown after the task
stop time. This is to remove any discontinuity between pilots just before goal
and pilots who had just reached goal at task stop time.

\begin{pg}
If a Conical ESS is used, then all pilots who have crossed into the cone before
or at the task stop time will be scored for being in goal.
\end{pg}

A fixed amount of points is subtracted from the time points of each pilot that
makes goal in a stopped task. This amount is the amount of time points a pilot
would receive if he had reached ESS exactly at the task stop time. This is to
remove any discontinuity between pilots just before ESS and pilots who had just
reached ESS at task stop time.
\begin{align*}
    typeOfTask = RaceToGoal \land numberOfStartGates = 1 : \\
    timePointsReduction &= timePoints(taskStopTime - startTime) \\
    \\
    typeOfTask \neq RaceToGoal \lor numberOfStartGates > 1 : \\
    timePointsReduction &= \\
    timePoints(taskStopTime - \max(\forall p : p \in StartedPilots : startTime_p)) \\
    \\
    \forall p : p \in PilotsInGoal : finalTimePoints_p &= timePoints_p - timePointsReduction
\end{align*}

\subsubsection{Distance points with altitude bonus}
\label{sec:distance-stopped-tasks}
To compensate for altitude differences at the time when a task is stopped,
a bonus distance is calculated for each point in the pilots’ track logs, based
on that point’s altitude above goal. This bonus distance is added to the
distance achieved at that point. All altitude values used for this calculation
are GPS altitude values, as received from the pilots’ GPS devices (no
compensation for different earth models applied by those devices). For all
distance point calculations, including the difficulty calculations in
hang-gliding (see ~\ref{sec:difficulty-calculation}), these new stopped
distance values are being used to determine the pilots’ best distance values.
Time and leading point calculations remain the same: they are not affected by
the altitude bonus or stopped distance values.
\begin{align*}
    \hgh{GlideRatio} &= 5.0 \\
    \pgh{GlideRatio} &= 4.0 \\
    \forall p : p \in PilotsLandedBeforeGoal : bestDistance_p &= max(minimumDistance, taskDistance - \min(xs)) \\
    \forall p : p \in PilotsReachedGoal : bestDistance_p &= taskDistance
\end{align*}
\[ xs = \forall track_p . point_i : shortestDistanceToGoal(track_p . point_i) - (track_p . point_i . altitude - GoalAltitude) * GlideRatio \\ \]

\subsection{Penalties}
Penalties for various actions are defined in the rules. These penalties are
either expressed as an absolute number (e.g. “100 points”) or as a percentage
(e.g. “10\% of the pilot’s score in the task where he performed the punishable
action”). The corresponding number of points is then deducted from the punished
pilot’s total score to calculate his final score.
\begin{align*}
    finalScore_p &= score_p - absolutePenalty \\
    finalScore_p &= score_p - (1 - percentagePenalty_p)
\end{align*}

\begin{hg}
These penalties are completely independent of any “Jump the Gun”-Penalty
a pilot may have incurred.
\end{hg}

The penalty mechanism can also be used to award bonus points to a pilot for
some actions like helping a pilot in distress. In that case the penalty must be
given as a negative number.

Any rounding up of scores is to be done after the application of penalties. The
lowest score a pilot can attain in a task, regardless of any incurred
penalties, is zero points.
\end{document}
