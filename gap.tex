% !TEX TS-program = xelatex
% !TEX encoding = UTF-8 Unicode
\documentclass{article}
\usepackage{civl}

% SEE: https://ctan.org/pkg/titlepages
\newcommand*{\titleGap}{\begingroup
\raggedleft
\vspace*{16\baselineskip}

\begin{tabularx}{\textwidth}{@{}rXr@{}}
    \textit{Fédération} & & \\
    \textit{Aéronautique} & & \multirow{2}{*}{} \\ \cline{2-3}
    \textit{Internationale} & & \\
\end{tabularx}

{\Huge CIVL GAP}\\
\vspace*{2\baselineskip}
{\Large Centralised Cross-Country Competition Scoring for}\\
{\Large Hang-Gliding and Paragliding}\\
\vspace*{2\baselineskip}
2016 Edition\\
Revision 1.1\\
Published February 26, 2017\\
\vspace*{10\baselineskip}

\begin{tabularx}{\textwidth}{@{}rXr@{}}
    \textit{Maison du Sport International} & & \\
    \textit{Av. de Rhodanie 54} & & \\
    \textit{CH-1007 Lausanne} & & \\
    \textit{(Switzerland)} & & \\
    \textit{Tél. +41 (0)21 345 10 70} & & \\
    \textit{Fax +41 (0)21 345 10 77} & & \\
    \textit{E-mail: \href{mailto:sec@fai.org}{sec@fai.org}} & & \\
    \textit{Web: \href{www.fai.org}{www.fai.org}} & & \\
\end{tabularx}

\endgroup}

\import{}{footer}
\pagestyle{gen}

\begin{document}
\import{}{header}

% SEE: http://tex.stackexchange.com/questions/86249/maketitle-text-before-title
{\let\newpage\relax\titleGap}\thispagestyle{pageone}

\newpage
Editor's note: Hang-gliding and paragliding are sports in which both men and women participate. Throughout this document the words "he", "him" or "his" are intended to apply equally to either sex unless it is specifically stated otherwise.\\

\begin{tabularx}{\textwidth}{|l|l|X|}
 \hline
    \textbf{Revision} & \textbf{Author} & \textbf{Changes} \\
 \hline
 2014 R1.1 & Joerg Ewald & Correction of a number of typos \\
 \hline
 2015 R1.0 & Joerg Ewald & Added changes from the 2015 CIVL Plenary:
    \begin{itemize}
        \item Use same leading coefficient calculation for both hang gliding and paragliding
        \item Use QNH for altitude measurement
        \item Final glide decelerators in paragliding no longer mandatory
        \item Distance measurement based on WGS84 ellipsoid postponed
    \end{itemize} \\
 \hline
 2016 R1.0 & Joerg Ewald & Added changes from the 2016 CIVL Plenary:
    \begin{itemize}
        \item PG: Double the amount of available leading points (reducing time points accordingly)
        \item PG: If no pilot in goal, the leading points weight is proportional to the task distance covered by pilots, up to 100 points.
        \item HG: Jump the gun penalty is now one point for every 2 seconds (was for every 3 seconds).
    \end{itemize} \\
 \hline
 2014 R1.1 & Joerg Ewald & Correct Figures 6 and 7 (point allocation graphs) \\
\hline
\end{tabularx}

\vspace*{\fill}
\section*{FEDERATION AERONAUTIQUE INTERNATIONALE}
\textbf{Maison du Sport International – Avenue de Rhodanie 54 – CH-1007 Lausanne – Switzerland}\\
Copyright 2014\\
All rights reserved. Copyright in this document is owned by the Fédération Aéronautique Internationale (FAI). Any person acting on behalf of the FAI or one of its Members is hereby authorised to copy, print, and distribute this document, subject to the following conditions:\\
\textbf{
    \begin{enumerate}
        \item The document may be used for information only and may not be exploited for commercial purposes.\\
        \item Any copy of this document or portion thereof must include this copyright notice.
    \end{enumerate}
}

\newpage
Note that any product, process or technology described in the document may be the subject of other Intellectual Property rights reserved by the Fédération Aéronautique Internationale or other entities and is not licensed hereunder.

\newpage
\tableofcontents

\newpage
\section{Introduction}
\subsection{Scope}
\subsection{Sources}
\subsection{Changes from previous edition}
\subsection{Differences between Hang-Gliding and Paragliding}

\newpage
\section{The GAP Philosophy}
\subsection{History}
\subsection{Scoring Process}

\newpage
\section{Definitions}
\subsection{Flights}
\subsection{Locations and distances}
\subsection{Times}

\newpage
\section{Use of Tracklog Data}
\subsection{Position}
\subsection{Distance}
\label{sec:distance}
In general, task evaluation occurs in the \(x/y\) plain, therefore distance
measurements are always exclusively horizontal measurements. The earth model used
is the FAI sphere, with a radius of 6371.0 km.

\colorbox{pgc}{\begin{minipage}{\textwidth}
In paragliding, for final glide decelerators (\ref{sec:final-glide-decelerators})
and altitude bonus in stopped tasks (\ref{sec:distance-stopped-tasks}), altitude
is also considered, but this does not affect distance calculations between two
geographic points.
\end{minipage}}
\marginpar{\includegraphics[scale=0.6]{pg.png}\hfill}

\colorbox{hgc}{\begin{minipage}{\textwidth}
In hang gliding, for altitude bonus in stopped tasks (\ref{sec:distance-stopped-tasks}),
altitude is also considered, but this does not affect distance calculations between
two geographic points.
\end{minipage}}
\marginpar{\includegraphics[scale=1.2]{hg.png}\hfill}

The distance between two points, identified by their radian coordinates
\(lat_1/long_1\) and \(lat_2/long_2\), is calculated using the Haversine formula.
\begin{align*}
    distLat &= lat_2 - lat_1 \\
    distLong &= long_2 - long_1 \\
    a &= \sin(\frac{distLat}{2})^2 + \cos lat_1 * \cos lat_2 * \sin(\frac{distLong}{2})^2 \\
    radianDistance &= 2 * \atantwo(\sqrt a, \sqrt{1 - a}) \\
    distance &= radianDistance * 6371000 \\
\end{align*}
To reproduce this formula in Excel, the following modification is necessary due to a different
implementation of the \(arctan2\) function:
\[ radianDistance = pi - 2 * \atantwo(\sqrt a, \sqrt{1 - a}) \]

\subsection{Altitude}
\subsection{Time}

\newpage
\section{Competition Parameters}
\subsection{Nominal Launch}
\subsection{Nominal Distance}
\subsection{Minimum Distance}
\subsection{Nominal Goal}
\subsection{Nominal Time}
\subsection{Final Glide Decelerator}
\label{sec:final-glide-decelerators}
\subsection{Score-back Time}

\newpage
\section{Task Setting}
\subsection{Definition of a task}
\subsubsection{Race task}
\subsubsection{Open distance task}
\subsection{Definition of control zones}
\subsubsection{Definition of a turnpoint cylinder}
\subsubsection{Definition of conical end of speed section}
\subsection{Definiton of goal}
\subsubsection{Goal line}
\subsection{Start procedures}
\subsubsection{Air start}
\subsubsection{Ground start}
\subsubsection{Race to goal}
\subsubsection{Elapsed time}
\subsection{Distances}
\subsubsection{Task distance}
\label{sec:task-distance}
\subsubsection{Speed section distance}

\newpage
\section{Flying a task}
\subsection{Race task}
\subsection{Open distance task}

\newpage
\section{Task evaluation}
\subsection{Reaching a control zone}
\subsubsection{Reaching a turnpoint cylinder}
A cylinder is considered “reached” by a pilot if that pilot’s track log shows the pilot crossing out of the
cylinder in the case of an exit cylinder, or into the cylinder in case of an enter cylinder, by containing at
least one track point closer to the cylinder’s centre than the cylinder radius (enter) or further away from
the cylinder’s centre than the cylinder radius (exit). During task evaluation, only the \(x/y\) coordinates are
considered, and a point must lie within (enter) or outside of (exit) the circle representing the turnpoint
cylinder in the \(x/y\) plain. This is determined by measuring the distance between a track point and the
turnpoint. This distance must be greater (exit) or smaller (enter) than the cylinder’s radius.
To compensate for different distance calculations and different earth models in use by today’s GPS
devices (FAI sphere vs. WGS84 ellipsoid), a 0.5\% tolerance is used for this calculation. This had to be
introduced so that a pilot reading the distance to the next cylinder centre from his GPS device can rely
on having reached the turnpoint when the distance displayed by the instrument is smaller than the
defined turnpoint cylinder radius.

For enter cylinders, this means that a tracklog point that is closer to the turnpoint than \(r*1.005\) is
considered proof of reaching the turnpoint. For exit cylinders, this means that a tracklog point that is
further away from the turnpoint than \(r*0.995\) is considered proof of reaching the turnpoint.
Starting January 1st, 2015 – provided all distance measurements are then based on the WGS84 ellipsoid
(see~\ref{sec:distance}) – the tolerance for turnpoint cylinders in CIVL’s FAI Category 1 events will be reduced to 0.01\%
of the cylinder radius, with a minimum of 5 meter. Organisers of FAI Category 2 events may continue
using the existing 0.5\% tolerance, to accommodate pilots flying with instruments which calculate
distance based on the FAI sphere.

The time when a control zone was reached is determined by the time a so-called “crossing” occurred. A
crossing is defined as a pair of consecutive track points, of which at least one lies inside the band
determined by the turnpoint’s centre, its radius and the tolerance value.
\begin{align*}
    tolerance &= 0.5\% | 0.01\% \\
    minTolerance &= 0m | 5m \\
    turpoint_i : innerRadius_i &= \min(radius_i * (1 - tolerance), radius_i - minTolerance) \\
    turpoint_i : outerRadius_i &= \max(radius_i * (1 + tolerance), radius_i + minTolerance) \\
\end{align*}
\begin{equation*}
    crossing_i = \exists_j (a \land b) \lor (c \land d)
\end{equation*}
\begin{align*}
    a = distance(center_i, trackpoint_j) &>= innerRadius_i \\
    b = distance(center_i, trackpoint_{j+1}) &<= outerRadius_i \\
    c = distance(center_i, trackpoint_j) &<= outerRadius_i \\
    d = distance(center_i, trackpoint_{j+1}) &>= innerRadius_i \\
\end{align*}
The time of a crossing depends on whether it actually cuts across the actual cylinder, or whether both
points lie within the tolerance band, but on the same side of the actual cylinder.
\begin{align*}
    crossing . time &= trackpoint_{j+1} . time \ when \ A \\
    crossing . time &= trackpoint_j . time \ when \ B \\
    crossing . time &= interpolateTime(trackpoint_{j+1}, trackpoint_{j+1}) . time \ when \ C \\
\end{align*}
\begin{align*}
    A &= (a \land b) \lor (c \land d) \land turnpoint_i = ESS \\
    B &= (a \land b) \lor (c \land d) \land turnpoint_i \neq ESS \\
    C &= (a \land d) \lor (c \land b) \\
    \\
    a = distance(center_i, trackpoint_j) &< radius_i \\
    b = distance(center_i, trackpoint_{j+1}) &< radius_i \\
    c = distance(center_i, trackpoint_j) &> radius_i \\
    d = distance(center_i, trackpoint_{j+1}) &> radius_i \\
\end{align*}
The method used to interpolate the crossing time is buried in FS’ code and will have to be documented
at a later point.

Finally, given all n crossings for a turnpoint cylinder, sorted in ascending order by their crossing time, the
time when the cylinder was reached is determined.
\begin{align*}
    turnpoint_i &= SSS : reachingTime_i = crossing_n . time \\
    turnpoint_i &\neq SSS : reachingTime_i = crossing_0 . time
\end{align*}
\subsection{Reaching a conical end of speed section}
\subsection{Reaching goal}
\subsubsection{Goal cylinder}
\subsubsection{Goal line}
\subsection{Flown distance}
\subsubsection{Race task}
To determine a pilot’s flown distance, a first step determines which turnpoints he reached considering
all timing restrictions: launch within launch time window, valid start, but only until the task deadline
time. After the last turnpoint the pilot reached, for every remaining track point, the shortest distance to
goal is calculated using the method described in section \ref{sec:task-distance}. The flown distance is then calculated as
task distance minus the shortest distance the pilot still had to fly. Therefore, for scoring, the pilot’s best
distance along the course line is considered, regardless of where the pilot landed in the end.

If a pilot flies less than minimum distance, he will be scored for this minimum distance. This also applies
to pilots who are not able to produce a valid GPS tracklog, but for whom launch officials verify launch
within the launch window.

If a pilot reaches goal, he will be scored for the task distance.
\begin{align*}
    \forall p : p \in PilotsLandingBeforeGoal : bestDistance_p &= \max(minimumDistance, taskDistance - shortest) \\
    shortest &= \min(\forall track_p . point_i shortestDistanceToGoal(track_p . point_i))) \\
    \forall p : p \in PilotsReachingGoal : bestDistance_p &= taskDistance
\end{align*}
\subsubsection{Open distance task}
\subsection{Time for speed section}

\newpage
\section{Task Validity}
\subsection{Launch Validity}
\subsection{Distance Validity}
\subsection{Time Validity}

\newpage
\section{Points Allocation}

\newpage
\section{Pilot score}
\subsection{Distance points}
\subsubsection{Difficulty calculation}
\subsubsection{Example for difficulty calculation}
\subsection{Time points}
\subsection{Leading points}
\subsubsection{Leading coefficient}
\subsubsection{Example}
\subsection{Arrival points}

\newpage
\section{Special cases}
\subsection{ESS but not goal}
\subsection{Early start}
\subsection{Stopped tasks}
\subsubsection{Stop task time}
\subsubsection{Requirements to score a stopped task}
\subsubsection{Stopped task validity}
\subsubsection{Scored time window}
\subsubsection{Time points for pilots at or after ESS}
\subsubsection{Distance points with altitude bonus}
\label{sec:distance-stopped-tasks}
\subsection{Penalties}

\newpage
\section{Task ranking}
\subsection{Overall task ranking}
\subsection{Female task ranking}
\subsection{Nation task ranking}

\newpage
\section{Competition ranking}
\subsection{Overall competition ranking}
\subsection{Female competition ranking}
\subsection{Nation competition ranking}
\subsection{Ties}

\newpage
\section{FTV - Fixed Total Validity}

\end{document}
